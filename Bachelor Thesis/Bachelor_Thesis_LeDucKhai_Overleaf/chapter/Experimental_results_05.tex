\section{Effectiveness of intermediate loss}


\subsection{Effectiveness of Intermediate Cross-Entropy Loss}

%% \usepackage[normalem]{ulem}
% \usepackage{multirow}


\begin{table}[!ht]
\centering
\begin{adjustbox}{width=\columnwidth,center}
\begin{tabular}{|c|c|c|c|c|c|c|c|} 
\hline
Arch.                              & Init                              & Pre-training data                                                                    & Recog. & Baseline      & +ICE          & +IF           & +L2           \\ 
\hline
\multirow{8}{*}{\textit{Large}\textsubscript{1-8}} & \multirow{2}{*}{\textit{XLSR-53}}          & \multirow{2}{*}{None}                                                                & dev    & \textbf{27.9} & 28.4          & 28.0          & -             \\ 
\cline{4-8}
                                   &                                   &                                                                                      & test   & \textbf{32.3} & 33.3          & 32.8          & -             \\ 
\cline{2-8}
                                   & \multirow{4}{*}{None}             & \multirow{2}{*}{\begin{tabular}[c]{@{}c@{}}In-house data Viet.\\(219h)\end{tabular}} & dev    & 30.4          & 29.1          & \textbf{28.6} & 28.8          \\ 
\cline{4-8}
                                   &                                   &                                                                                      & test   & 33.4          & 33.7          & \textbf{33.0} & \uline{32.9}  \\ 
\cline{3-8}
                                   &                                   & \multirow{2}{*}{None}                                                                & dev    & 35.6          & 33.8          & \textbf{33.0} & \uline{31.4}  \\ 
\cline{4-8}
                                   &                                   &                                                                                      & test   & 40.7          & \textbf{38.1} & 38.8          & \uline{36.4}  \\ 
\cline{2-8}
                                   & \multirow{2}{*}{\textit{XLSR-53}} & \multirow{4}{*}{\begin{tabular}[c]{@{}c@{}}In-house data Viet.\\(219h)\end{tabular}} & dev    & 25.5          & 25.2          & \textbf{24.7} & \uline{24.3}  \\ 
\cline{4-8}
                                   &                                   &                                                                                      & test   & \textbf{29.1} & 29.2          & \textbf{29.1} & 29.3          \\ 
\cline{1-2}\cline{4-8}
\multirow{2}{*}{\textit{Base}}     & \multirow{4}{*}{None}             &                                                                                      & dev    & 30.2          & 29.7          & \textbf{29.0} & \uline{28.8}  \\ 
\cline{4-8}
                                   &                                   &                                                                                      & test   & 33.3          & 33.4          & \textbf{33.0} & \uline{32.3}  \\ 
\cline{1-1}\cline{3-8}
\multirow{2}{*}{\textit{Large}\textsubscript{1-8}} &                                   & \multirow{2}{*}{\begin{tabular}[c]{@{}c@{}}Viet. YT\\(1168h)\end{tabular}}           & dev    & 29.8          & 27.3          & \textbf{26.1} & -             \\ 
\cline{4-8}
                                   &                                   &                                                                                      & test   & 35.2          & 31.5          & \textbf{30.8} & -             \\
\hline
\end{tabular}
\end{adjustbox}
\caption{
    Comparison of \glspl{WER} between architecture sizes, pretrained datasets and pretrained models, all are finetuned on our in-house spontaneous telephone speech dataset and recognized at epoch 33 on dev and test telephone HYKIST data. 
    In this table, only 1 intermediate layer is applied in the middle \glsxtrshort{Transformer} block, e.g. position 4 for \textit{Large}\textsubscript{1-8} and 6 for \textit{Base} architecture. 
    \textbf{Bold} numbers are used to compare between the baseline and 2 types of intermediate loss, while \underline{underline} numbers denote extra improvement of L2 regularization compared to \glsxtrshort{IF Loss}}
\label{int_and_focal_WERs}
\end{table}
%\begin{table}[!ht]
\centering
\begin{adjustbox}{width=\columnwidth,center}
\begin{tabular}{|c|c|c|c|c|c|c|} 
\hline
Arch.                                                                      & Init                              & Pre-training data                                                                    & Recog. & Baseline      & +ICE           & +IF            \\ 
\hline
\multirow{12}{*}{\begin{tabular}[c]{@{}c@{}}\\\textit{Large}\textsubscript{1-8}\end{tabular}} & \multirow{3}{*}{\textit{XLSR-53}} & \multirow{3}{*}{None}                                                                & dev    & \textbf{14.8} & 15.8           & 15.4           \\ 
\cline{4-7}
                                                                           &                                   &                                                                                      & test   & \textbf{32.5} & 33.9           & 33.6           \\ 
\cline{4-7}
                                                                           &                                   &                                                                                      & vivos  & 30.3          & \textbf{30.0}  & \textbf{30.0}  \\ 
\cline{2-7}
                                                                           & \multirow{6}{*}{None}             & \multirow{3}{*}{In-house data Viet.(219h)}                                           & dev    & 16.4          & 16.1           & \textbf{15.8}  \\ 
\cline{4-7}
                                                                           &                                   &                                                                                      & test   & 35.6          & 34.8           & \textbf{34.5}  \\ 
\cline{4-7}
                                                                           &                                   &                                                                                      & vivos  & 31.3          & 30.4           & \textbf{29.6}  \\ 
\cline{3-7}
                                                                           &                                   & \multirow{3}{*}{None}                                                                & dev    & 20.8          & \textbf{18.6}  & 19.7           \\ 
\cline{4-7}
                                                                           &                                   &                                                                                      & test   & 44.7          & \textbf{42.1}  & 43.1           \\ 
\cline{4-7}
                                                                           &                                   &                                                                                      & vivos  & 34.9          & \textbf{33.1}  & 33.9           \\ 
\cline{2-7}
                                                                           & \multirow{3}{*}{\textit{XLSR-53}} & \multirow{6}{*}{\begin{tabular}[c]{@{}c@{}}In-house data Viet.\\(219h)\end{tabular}} & dev    & \textbf{11.5} & 12.3           & 13.0           \\ 
\cline{4-7}
                                                                           &                                   &                                                                                      & test   & \textbf{29.4} & 29.8           & 29.8           \\ 
\cline{4-7}
                                                                           &                                   &                                                                                      & vivos  & \textbf{27.2} & 27.7           & 27.6           \\ 
\cline{1-2}\cline{4-7}
\multirow{3}{*}{\textit{Base}}                                             & \multirow{6}{*}{None}             &                                                                                      & dev    & 16.6          & \textbf{15.4}~ & 15.9           \\ 
\cline{4-7}
                                                                           &                                   &                                                                                      & test   & 35.4          & \textbf{34.2}  & 34.4           \\ 
\cline{4-7}
                                                                           &                                   &                                                                                      & vivos  & 30.9          & 31.3           & \textbf{30.5}  \\ 
\cline{1-1}\cline{3-7}
\multirow{3}{*}{\begin{tabular}[c]{@{}c@{}}\\\textit{Large}\textsubscript{1-8}\end{tabular}}  &                                   & \multirow{3}{*}{Viet. YT(1168h)}                                                     & dev    & 16.4          & 15.6           & \textbf{14.5}  \\ 
\cline{4-7}
                                                                           &                                   &                                                                                      & test   & 34.4          & 32.2           & \textbf{30.9}  \\ 
\cline{4-7}
                                                                           &                                   &                                                                                      & vivos  & 28.7          & 27.6           & \textbf{26.9}  \\
\hline
\end{tabular}
\end{adjustbox}
\caption{
    Recognition \glspl{WER} at epoch 33 on read speech datasets. 
    Experiment setups are the same to Table \ref{int_and_focal_WERs}. 
    The recognition is done on CommonVoice dev/test sets and VIVOS test set.}
\label{int_and_focal_WERs_CV_Vivos}
\end{table}

\textbf{Improvement on HYKIST data}: In Table \ref{table:int_loss_hykist_pos}, when using in-house telephone dataset to train and transcribe the HYKIST dataset with the help of \glsxtrshort{ICE Loss}, we report the total improvement in performance for from scratch experiment where the \Glspl{WER} decrease from 35.6\% and 40.7\% to 33.8\% and 38.1\% on dev and test set respectively.
For \glsxtrshort{YT} experiment, the \Glspl{WER} decrease from 29.8\% and 35.2\% to 27.3\% and 31.5\%.
We also report a small improvement for the combination of Vietnamese in-house data and \glsxtrshort{YT} data, from 25.3\% and 27.2\% to 25.1\% and 27.1\%.

% \usepackage[normalem]{ulem}
% \usepackage{multirow}


\begin{table}[!ht]
\captionsetup{font=Large}
\centering
\begin{adjustbox}{width=0.8\columnwidth,center}
\begin{tabular}{|c|c|c|c|c|} 
\hline
\multicolumn{2}{|c|}{Pre-training}                           & \multirow{2}{*}{With ICE} & \multicolumn{2}{c|}{WER [\%]}  \\ 
\cline{1-2}\cline{4-5}
Data                                 & Hours                 &                           & Hykist dev & Hykist test       \\ 
\hline
\multirow{2}{*}{None}                & \multirow{2}{*}{-}    & No                        & 35.6       & 40.7              \\ 
\cline{3-5}
                                     &                       & Yes                       & 33.8       & 38.1              \\ 
\hline
\multirow{2}{*}{Viet. YT}            & \multirow{4}{*}{1168} & No                        & 29.8       & 35.2              \\ 
\cline{3-5}
                                     &                       & Yes                       & 27.3       & 31.5              \\ 
\cline{1-1}\cline{3-5}
\multirow{2}{*}{Viet. in-house + YT} &                       & No                        & 25.3       & 27.2              \\ 
\cline{3-5}
                                     &                       & Yes                       & 25.1       & 27.1              \\
\hline
\end{tabular}
\end{adjustbox}
\caption{
    \center{Improvements of WERs on HYKIST data between pretraining schedules when applying ICE Loss. 
    All pretrainings use the \textit{Large}\textsubscript{1-8} architecture with random initialization and are finetuned until full convergence on Vietnamese in-house data. 
    Only 1 intermediate layer is applied in the middle Transformer block, e.g. position 4 for \textit{Large}\textsubscript{1-8} and 6 for \textit{Base} architecture.}}
\label{table:int_loss_hykist_pos}
\end{table}

\textbf{Degradation on HYKIST data}: As shown in Table \ref{table:int_loss_hykist_neg}, for the directly finetuning experiment with \glsxtrshort{XLSR-53} preloaded, the performance is hurt totally (both \Glspl{WER} on dev and test sets increase).
Besides, both continued pretrainings on Vietnamese in-house and on \glsxtrshort{YT} data experience the partial improvements (only \Glspl{WER} on test sets are slightly increased but \Glspl{WER} on dev sets decrease).
The rest pretraining schedules in Table \ref{table:int_loss_hykist_neg} also experience partial improvements.

% \usepackage[normalem]{ulem}
% \usepackage{multirow}


\begin{table}[!ht]
\centering
\begin{adjustbox}{width=\columnwidth,center}
\begin{tabular}{|c|c|c|c|c|c|} 
\hline
\multirow{2}{*}{Arch.}         & \multirow{2}{*}{Init.}             & \multirow{2}{*}{Pre-training data}                                                    & \multirow{2}{*}{With ICE} & \multicolumn{2}{c|}{WER [\%]}  \\ 
\cline{5-6}
                               &                                    &                                                                                       &                           & Hykist dev & Hykist test       \\ 
\hline
\multirow{6}{*}{\textit{Large}\textsubscript{1-8}}      & \multirow{2}{*}{\textit{XLSR-53}}  & \multirow{2}{*}{None}                                                                 & No                        & 27.9       & 32.3              \\ 
\cline{4-6}
                               &                                    &                                                                                       & Yes                       & 28.4       & 33.3              \\ 
\cline{2-6}
                               & \multirow{2}{*}{None}              & \multirow{6}{*}{\begin{tabular}[c]{@{}c@{}}Viet. in-house\\(219h)\end{tabular}}       & No                        & 30.4       & 33.4              \\ 
\cline{4-6}
                               &                                    &                                                                                       & Yes                       & 29.1       & 33.7              \\ 
\cline{2-2}\cline{4-6}
                               & \multirow{2}{*}{\textit{XLSR-53 }} &                                                                                       & No                        & 25.5       & 29.1              \\ 
\cline{4-6}
                               &                                    &                                                                                       & Yes                       & 25.2       & 29.2              \\ 
\cline{1-2}\cline{4-6}
\multirow{2}{*}{\textit{Base}} & \multirow{4}{*}{None}              &                                                                                       & No                        & 30.2       & 33.3              \\ 
\cline{4-6}
                               &                                    &                                                                                       & Yes                       & 29.7       & 33.4              \\ 
\cline{1-1}\cline{3-6}
\multirow{4}{*}{\textit{Large}\textsubscript{1-8}}      &                                    & \multirow{2}{*}{\begin{tabular}[c]{@{}c@{}}Multiling. in-house\\(1168h)\end{tabular}} & No                        & 26.8       & 28.7              \\ 
\cline{4-6}
                               &                                    &                                                                                       & Yes                       & 25.5       & 29.4              \\ 
\cline{2-6}
                               & \multirow{2}{*}{\textit{XLSR-53}}  & \multirow{2}{*}{\begin{tabular}[c]{@{}c@{}}Viet. YT\\(1168h)\end{tabular}}            & No                        & 24.3       & 28.1              \\ 
\cline{4-6}
                               &                                    &                                                                                       & Yes                       & 23.7       & 28.2              \\
\hline
\end{tabular}
\end{adjustbox}
\caption{
    Degradations of \glspl{WER} {[}\%{]} on HYKIST data between pretraining schedules when applying \glsxtrshort{ICE Loss}. All models are finetuned until full convergence on Vietnamese in-house data. 
    Only 1 intermediate layer is applied in the middle \glsxtrshort{Transformer} block, e.g. position 4 for \textit{Large}\textsubscript{1-8} and 6 for \textit{Base} architecture.
    }
\label{table:int_loss_hykist_neg}
\end{table}

\textbf{Improvement on CommonVoice and VIVOS data}: In the situation of more out-of-domain recognition shown in Table \ref{int_loss_cvvivos_pos}, which means using the model finetuned on our in-house spontaneous telephone speech dataset to do the recognition on read speech datasets like CommonVoice and VIVOS, we report the total improvements in performance for \textit{Large}\textsubscript{1-8} in-house pretraining, from scratch and \glsxtrshort{YT} experiments. 
Notable is from scratch training where \glsxtrshort{WER}s reduce from 20.8\%, 44.7\%, 34.9\% to 18.6\%, 42.1\%, 33.1\%; and \glsxtrshort{YT} pretraining where \glsxtrshort{WER}s reduce from 16.4\%, 34.4\%, 28.7\% to 15.6\%, 32.2\%, 27.6\% on CommonVoice dev/test and VIVOS test set respectively.
Together with the improvements on HYKIST reported in Table \ref{table:int_loss_hykist_pos}, we conclude that using \glsxtrshort{ICE Loss} for from scratch training and for pretraining on \glsxtrshort{YT} data improves the recognitions on both telephone and read speech domain.

\begin{table}[!ht]
\captionsetup{font=Large}
\centering
\begin{adjustbox}{width=0.7\columnwidth,center}
\begin{tabular}{|c|c|c|c|c|c|} 
\hline
\multicolumn{2}{|c|}{Pre-training}                      & \multirow{2}{*}{With ICE} & \multicolumn{3}{c|}{WER [\%]}  \\ 
\cline{1-2}\cline{4-6}
Data                            & Hours                 &                           & CV dev & CV test & Vivos       \\ 
\hline
\multirow{2}{*}{None}           & \multirow{2}{*}{-}    & No                        & 20.8   & 44.7    & 34.9        \\ 
\cline{3-6}
                                &                       & Yes                       & 18.6   & 42.1    & 33.1        \\ 
\hline
\multirow{2}{*}{Viet. YT}       & \multirow{2}{*}{1168} & No                        & 16.4   & 34.4    & 28.7        \\ 
\cline{3-6}
                                &                       & Yes                       & 15.6   & 32.2    & 27.6        \\ 
\hline
\multirow{2}{*}{Viet. in-house} & \multirow{2}{*}{219}  & No                        & 16.4   & 35.6    & 31.3        \\ 
\cline{3-6}
                                &                       & Yes                       & 16.1   & 34.8    & 30.4        \\
\hline
\end{tabular}
\end{adjustbox}
\caption{
    \center{Improvements of WERs on CommonVoice and VIVOS between pretraining schedules when applying ICE Loss.
    All pretrainings use the \textit{Large}\textsubscript{1-8} architecture with random initialization and are finetuned until full convergence on Vietnamese in-house data.}}
\label{int_loss_cvvivos_pos}
\end{table}

\textbf{Degradation on CommonVoice and VIVOS data}: As shown in the Table \ref{int_loss_cvvivos_neg}, we experience the total degradations for 2 cases: continued pretraining on Vietnamese in-house data and pretraining on the combination of in-house and \glsxtrshort{YT} data; where \glsxtrshort{WER}s for all read speech test sets increase.
The rest cases experience partial degradations.

\begin{table}[!ht]
\captionsetup{font=Large}
\centering
\begin{adjustbox}{width=0.8\columnwidth,center}
\begin{tabular}{|c|c|c|c|c|c|c|c|} 
\hline
\multirow{2}{*}{Arch.}             & \multirow{2}{*}{Init.}            & \multicolumn{2}{c|}{Pre-training}                            & \multirow{2}{*}{With ICE} & \multicolumn{3}{c|}{WER [\%]}  \\ 
\cline{3-4}\cline{6-8}
                                   &                                   & Data                                 & Hours                 &                           & CV dev & CV test & Vivos       \\ 
\hline
\multirow{4}{*}{\textit{Large}\textsubscript{1-8}} & \multirow{4}{*}{\textit{XLSR-53}} & \multirow{2}{*}{None}                & \multirow{2}{*}{-}    & No                        & 14.8   & 32.5    & 30.3        \\ 
\cline{5-8}
                                   &                                   &                                      &                       & Yes                       & 15.8   & 33.9    & 30.0        \\ 
\cline{3-8}
                                   &                                   & \multirow{4}{*}{Viet. in-house}      & \multirow{4}{*}{219}  & No                        & 11.5   & 29.4    & 27.2        \\ 
\cline{5-8}
                                   &                                   &                                      &                       & Yes                       & 12.3   & 29.8    & 27.7        \\ 
\cline{1-2}\cline{5-8}
\multirow{2}{*}{\textit{Base}}     & \multirow{6}{*}{None}             &                                      &                       & No                        & 16.6   & 35.4    & 30.9        \\ 
\cline{5-8}
                                   &                                   &                                      &                       & Yes                       & 15.4   & 34.2    & 31.3        \\ 
\cline{1-1}\cline{3-8}
\multirow{6}{*}{\textit{Large}\textsubscript{1-8}} &                                   & \multirow{2}{*}{Multiling. in-house} & \multirow{6}{*}{1168} & No                        & 15.2   & 29.7    & 29.5        \\ 
\cline{5-8}
                                   &                                   &                                      &                       & Yes                       & 14.8   & 30.5    & 28.8        \\ 
\cline{3-3}\cline{5-8}
                                   &                                   & \multirow{2}{*}{Viet. in-house + YT} &                       & No                        & 12.9   & 26.5    & 21.0        \\ 
\cline{5-8}
                                   &                                   &                                      &                       & Yes                       & 13.6   & 28.2    & 21.9        \\ 
\cline{2-3}\cline{5-8}
                                   & \multirow{2}{*}{\textit{XLSR-53}} & \multirow{2}{*}{Viet. YT}            &                       & No                        & 11.8   & 28.4    & 25.6        \\ 
\cline{5-8}
                                   &                                   &                                      &                       & Yes                       & 12.3   & 28.3    & 25.0        \\
\hline
\end{tabular}
\end{adjustbox}
\caption{
    \center{Degradations of WERs on CommonVoice and VIVOS between pretraining schedules when applying ICE Loss.}}
\label{int_loss_cvvivos_neg}
\end{table}

\bigskip

\subsection{Effectiveness of Intermediate Focal Loss}

\textbf{Effectiveness on HYKIST data}: 

As shown in Table \ref{table:if_loss_hykist_pos} and Table \ref{table:if_loss_hykist_neg} below, when using \glsxtrshort{IF Loss}, we see the \Glspl{WER} on HYKIST improved compared to the baselines for various pretraining schedules (7/9 experiments experience total improvements), compared to only 3/9 experiments experiencing total improvements using \glsxtrshort{ICE Loss} (\glsxtrshort{ICE Loss} results are shown in Table \ref{table:int_loss_hykist_pos} and Table \ref{table:int_loss_hykist_neg} above).
In addition, we report all \Glspl{WER} of \glsxtrshort{IF Loss} experiments to be lower than those of \glsxtrshort{ICE Loss} experiments, except the one on HYKIST test set of from scratch training. 
We therefore conclude that, when finetuning and recognizing on the same telephone domain, \glsxtrshort{IF Loss} works better than \glsxtrshort{ICE Loss}. 

Compared to our strongest continued pretraining baseline, the application of \glsxtrshort{IF Loss} on the combination of Vietnamese in-house and \glsxtrshort{YT} data (24.5\% and 27.1\%) outperforms the results of continued pretraining on the combination of Vietnamese in-house and \glsxtrshort{YT} data (24.5\% and 27.2\% on dev and test set respectively as shown in Table \ref{table: xlsr53_init_pretrain}).
However, we believe that the \glsxtrshort{IF Loss} can further reduce \glsxtrshort{WER}s for this continued pretraining schedule, as it does with continued pretraining on \glsxtrshort{YT} data. 

Among all total improvements reported in Table \ref{table:if_loss_hykist_pos}, notable is the \Glspl{WER} reduction of \glsxtrshort{YT} experiment from 29.8\% and 35.2\% to 26.1\% and 30.8\% on dev and test set respectively, whose relative \glsxtrshort{WERR} is around 12.5\% in average.
For a more diverse pretrained data (Vietnamese in-house data), we report the \glsxtrshort{WER}s reduction from 30.4\% and 33.4\% to 28.6\% and 33.0\%, whose relative \glsxtrshort{WERR} is around 3.6\% in average.
For even more diverse pretrained data (Vietnamese in-house  + \glsxtrshort{YT} data), we report the \glsxtrshort{WER}s reduction from 25.3\% and 27.2\% to 24.5\% and 27.1\%, whose relative \glsxtrshort{WERR} is around 1.8\% in average.
We therefore conclude that the effectiveness of \glsxtrshort{IF Loss} decreases when the pretrained data becomes more diverse.


% \usepackage[normalem]{ulem}
% \usepackage{multirow}


\begin{table}[!ht]
\centering
\begin{adjustbox}{width=\columnwidth,center}
\begin{tabular}{|c|c|c|c|c|c|} 
\hline
\multirow{2}{*}{Arch.}             & \multirow{2}{*}{Init.}             & \multirow{2}{*}{Pre-training data}                                                    & \multirow{2}{*}{With IF} & \multicolumn{2}{c|}{WER [\%]}  \\ 
\cline{5-6}
                                   &                                    &                                                                                       &                          & Hykist dev & Hykist test       \\ 
\hline
\multirow{6}{*}{\textit{Large}\textsubscript{1-8}} & \multirow{4}{*}{None}              & \multirow{2}{*}{None}                                                                 & No                       & 35.6       & 40.7              \\ 
\cline{4-6}
                                   &                                    &                                                                                       & Yes                      & 33.0       & 38.8              \\ 
\cline{3-6}
                                   &                                    & \multirow{6}{*}{\begin{tabular}[c]{@{}c@{}}Viet. in-house\\(219h)\end{tabular}}       & No                       & 30.4       & 33.4              \\ 
\cline{4-6}
                                   &                                    &                                                                                       & Yes                      & 28.6       & 33.0              \\ 
\cline{2-2}\cline{4-6}
                                   & \multirow{2}{*}{\textit{XLSR-53 }} &                                                                                       & No                       & 25.5       & 29.1              \\ 
\cline{4-6}
                                   &                                    &                                                                                       & Yes                      & 24.7       & 29.1              \\ 
\cline{1-2}\cline{4-6}
\multirow{2}{*}{\textit{Base}}     & \multirow{6}{*}{None}              &                                                                                       & No                       & 30.2       & 33.3              \\ 
\cline{4-6}
                                   &                                    &                                                                                       & Yes                      & 29.0       & 33.0              \\ 
\cline{1-1}\cline{3-6}
\multirow{6}{*}{\textit{Large}\textsubscript{1-8}} &                                    & \multirow{2}{*}{\begin{tabular}[c]{@{}c@{}}Viet. YT\\(1168h)\end{tabular}}            & No                       & 29.8       & 35.2              \\ 
\cline{4-6}
                                   &                                    &                                                                                       & Yes                      & 26.1       & 30.8              \\ 
\cline{3-6}
                                   &                                    & \multirow{2}{*}{\begin{tabular}[c]{@{}c@{}}Viet. in-house + YT\\(1168h)\end{tabular}} & No                       & 25.3       & 27.2              \\ 
\cline{4-6}
                                   &                                    &                                                                                       & Yes                      & 24.5       & 27.1              \\ 
\cline{2-6}
                                   & \multirow{2}{*}{\textit{XLSR-53}}  & \multirow{2}{*}{\begin{tabular}[c]{@{}c@{}}Viet. YT\\(1168h)\end{tabular}}            & No                       & 24.3       & 28.1              \\ 
\cline{4-6}
                                   &                                    &                                                                                       & Yes                      & 23.4       & 28.1              \\
\hline
\end{tabular}
\end{adjustbox}
\caption{
    Improvements of \glspl{WER} {[}\%{]} on HYKIST data between pretraining schedules when applying \glsxtrshort{IF Loss}. All models are finetuned until full convergence on Vietnamese in-house data. 
    Only 1 intermediate layer is applied in the middle \glsxtrshort{Transformer} block, e.g. position 4 for \textit{Large}\textsubscript{1-8} and 6 for \textit{Base} architecture.}
\label{table:if_loss_hykist_pos}
\end{table}

As shown in Table \ref{table:if_loss_hykist_neg}, only for the case of directly finetuning with \glsxtrshort{XLSR-53}, using \glsxtrshort{IF Loss} makes the \Glspl{WER} on HYKIST increased compared to the baselines. 
However, the degradation is rather small, from 27.9\% and 32.3\% to 28.0\% and 32.8\% on dev and test set respectively.
Besides, a partial degradation of performance is reported in the multilingual in-house data experiment, where the  average \glsxtrshort{WER} of dev and test set (25.2\% and 29.3\%) is even lower than the baseline (26.8\% and 28.7\%).
Hence, in a rapid deployment of an \glsxtrshort{ASR} system, we recommend the direct use of \glsxtrshort{IF Loss} in training without the need of one more training as a baseline for performance comparison.

% \usepackage[normalem]{ulem}
% \usepackage{multirow}


\begin{table}[!ht]
\captionsetup{font=Large}
\centering
\begin{adjustbox}{width=0.7\columnwidth,center}
\begin{tabular}{|c|c|c|c|c|c|} 
\hline
\multirow{2}{*}{Init.}            & \multicolumn{2}{c|}{Pre-training}                            & \multirow{2}{*}{With IF} & \multicolumn{2}{c|}{WER [\%]}  \\ 
\cline{2-3}\cline{5-6}
                                  & Data                                 & Hours                 &                          & Hykist dev & Hykist test       \\ 
\hline
\multirow{2}{*}{\textit{XLSR-53}} & \multirow{2}{*}{None}                & \multirow{2}{*}{-}    & No                       & 27.9       & 32.3              \\ 
\cline{4-6}
                                  &                                      &                       & Yes                      & 28.0       & 32.8              \\ 
\hline
\multirow{2}{*}{None}             & \multirow{2}{*}{Multiling. in-house} & \multirow{2}{*}{1168} & No                       & 26.8       & 28.7              \\ 
\cline{4-6}
                                  &                                      &                       & Yes                      & 25.2       & 29.3              \\
\hline
\end{tabular}
\end{adjustbox}
\caption{
    \center{Degradations of WERs on HYKIST data between pretraining schedules when applying IF Loss.
    All pretrainings use the \textit{Large}\textsubscript{1-8} architecture and are finetuned until full convergence on Vietnamese in-house data.}}
\label{table:if_loss_hykist_neg}
\end{table}

\bigskip

\textbf{Effectiveness on CommonVoice and VIVOS data}: 

In the larger domain-shift recognition, we still receive the significant reduction of \Glspl{WER} in multiple experiments as shown in Table \ref{table:if_loss_cvvivos_pos}. 
The notable reduction of \Glspl{WER} compared to baselines is again on \glsxtrshort{YT} data, whose \Glspl{WER}s decrease from 16.4\%, 34.4\% and 28.7\% to 14.5\%, 30.9\% and 26.9\% respectively for 3 read speech sets, that makes \glsxtrshort{WERR} about 9.3\% in average.
The \glsxtrshort{ICE Loss} in Table \ref{int_loss_cvvivos_pos} makes 3 experiments totally improved, while the \glsxtrshort{IF Loss} makes 4. 
Furthermore, the \glsxtrshort{WER}s for \glsxtrshort{IF Loss} on 3 read speech datasets are as competitive as \glsxtrshort{ICE Loss}.
In addition, when finetuning and recognizing on the same telephone domain, \glsxtrshort{IF Loss} works better than \glsxtrshort{ICE Loss} as proved above.
We therefore conclude that \glsxtrshort{IF Loss} works better than \glsxtrshort{ICE Loss} in all domains.

\begin{table}[!ht]
\captionsetup{font=Large}
\centering
\begin{adjustbox}{width=0.8\columnwidth,center}
\begin{tabular}{|c|c|c|c|c|c|c|} 
\hline
\multirow{2}{*}{Arch.}             & \multicolumn{2}{c|}{Pre-training}                       & \multirow{2}{*}{With IF} & \multicolumn{3}{c|}{WER [\%]}  \\ 
\cline{2-3}\cline{5-7}
                                   & Data                            & Hours                 &                          & CV dev & CV test & Vivos       \\ 
\hline
\multirow{4}{*}{\textit{Large}\textsubscript{1-8}} & \multirow{2}{*}{Viet. in-house} & \multirow{2}{*}{219}  & No                       & 16.4   & 35.6    & 31.3        \\ 
\cline{4-7}
                                   &                                 &                       & Yes                      & 15.8   & 34.5    & 29.6        \\ 
\cline{2-7}
                                   & \multirow{2}{*}{None}           & \multirow{2}{*}{-}    & No                       & 20.8   & 44.7    & 34.9        \\ 
\cline{4-7}
                                   &                                 &                       & Yes                      & 19.7   & 43.1    & 33.9        \\ 
\hline
\multirow{2}{*}{\textit{Base}}     & \multirow{2}{*}{Viet. in-house} & \multirow{2}{*}{219}  & No                       & 16.6   & 35.4    & 30.9        \\ 
\cline{4-7}
                                   &                                 &                       & Yes                      & 15.9   & 34.4    & 30.5        \\ 
\hline
\multirow{2}{*}{\textit{Large}\textsubscript{1-8}} & \multirow{2}{*}{Viet. YT}       & \multirow{2}{*}{1168} & No                       & 16.4   & 34.4    & 28.7        \\ 
\cline{4-7}
                                   &                                 &                       & Yes                      & 14.5   & 30.9    & 26.9        \\
\hline
\end{tabular}
\end{adjustbox}
\caption{
    \center{Improvements of WERs on CommonVoice and VIVOS between pretraining schedules when applying IF Loss.
    All models are finetuned until full convergence on Vietnamese in-house data.}}
\label{table:if_loss_cvvivos_pos}
\end{table}

However, in the larger domain-shift recognition, we still meet degradations of performance in experiments pretrained on diverse data, as shown in Table \ref{table:if_loss_cvvivos_neg}.
We therefore recommend the use of \glsxtrshort{IF Loss} only for less diverse pretrained data if the domain of finetuning and recognition data are too different.

\begin{table}[!ht]
\centering
\begin{adjustbox}{width=0.9\columnwidth,center}
\begin{tabular}{|c|c|c|c|c|c|} 
\hline
\multirow{2}{*}{Init.}            & \multirow{2}{*}{Pre-training data}                                                    & \multirow{2}{*}{With IF} & \multicolumn{3}{c|}{WER [\%]}  \\ 
\cline{4-6}
                                  &                                                                                       &                          & CV dev & CV test & Vivos       \\ 
\hline
\multirow{4}{*}{\textit{XLSR-53}} & \multirow{2}{*}{None}                                                                 & No                       & 14.8   & 32.5    & 30.3        \\ 
\cline{3-6}
                                  &                                                                                       & Yes                      & 15.4   & 33.6    & 30.0        \\ 
\cline{2-6}
                                  & \multirow{2}{*}{\begin{tabular}[c]{@{}c@{}}Viet. in-house\\(219h)\end{tabular}}       & No                       & 11.5   & 29.4    & 27.2        \\ 
\cline{3-6}
                                  &                                                                                       & Yes                      & 13.0   & 29.8    & 27.6        \\ 
\hline
\multirow{4}{*}{None}             & \multirow{2}{*}{\begin{tabular}[c]{@{}c@{}}Multiling. in-house\\(1168h)\end{tabular}} & No                       & 15.2   & 29.7    & 29.5        \\ 
\cline{3-6}
                                  &                                                                                       & Yes                      & 14.5   & 30.6    & 28.1        \\ 
\cline{2-6}
                                  & \multirow{2}{*}{\begin{tabular}[c]{@{}c@{}}Viet. in-house + YT\\(1168h)\end{tabular}} & No                       & 12.9   & 26.5    & 21.0        \\ 
\cline{3-6}
                                  &                                                                                       & Yes                      & 12.7   & 28.6    & 22.1        \\ 
\hline
\multirow{2}{*}{\textit{XLSR-53}} & \multirow{2}{*}{\begin{tabular}[c]{@{}c@{}}Viet. YT\\(1168h)\end{tabular}}            & No                       & 11.8   & 28.4    & 25.6        \\ 
\cline{3-6}
                                  &                                                                                       & Yes                      & 13.2   & 29.1    & 24.5        \\
\hline
\end{tabular}
\end{adjustbox}
\caption{
    Degradations of \glspl{WER} {[}\%{]} on CommonVoice and VIVOS between pretraining schedules when applying \glsxtrshort{IF Loss}. All models are finetuned on Vietnamese in-house data. 
    Only 1 intermediate layer is applied in the middle \glsxtrshort{Transformer} block, e.g. position 4 for \textit{Large}\textsubscript{1-8} and 6 for \textit{Base} architecture.}
\label{table:if_loss_cvvivos_neg}
\end{table}
