\section{Lexicon and language model}

% Please add the following required packages to your document preamble:
% \usepackage{multirow}
\begin{table}[!ht]
\setlength{\tabcolsep}{0.9em}
\centering
\begin{tabular}{|c|c|c|c|c|c|} 
\hline
\# words & vocab & \multicolumn{2}{c|}{dev} & \multicolumn{2}{c|}{test}  \\ 
\hline
in train & size  & OOV   & PPL              & OOV   & PPL                \\ 
\hline
500M     & 11k   & 0.1\% & 67               & 0.2\% & 69                 \\
\hline
\end{tabular}
\caption{ 4-gram LM for Vietnamese.}
\label{table:LM_stats}
\end{table}


\subsection{Lexicon}

The Babel project\footnote{\href{https://www.iarpa.gov/research-programs/babel}{https://www.iarpa.gov/research-programs/babel}} provided us the initial lexicon for the Vietnamese language.
The training lexicon is then created by extending the initial lexica with the toolkit Sequitur Grapheme-To-Phoneme\footnote{\href{https://github.com/sequitur-g2p/sequitur-g2p}{https://github.com/sequitur-g2p/sequitur-g2p}} \cite{bisani2008g2p}.
We supplement the lexicon with medical terms provided by our project partner Triaphon in order to decode the HYKIST data.
The final recognition lexica for Vietnamese are 11k in size as shown in Table \ref{table:LM_stats}.


\subsection{Language model}

\glspl{LM} used are 4-grams and use entire words. 
We create our \glspl{LM} using the training pipeline from the SRILM toolkit \cite{stolcke2002srilm}.
The first step is to create a \glsxtrshort{LM} for each monolingual text corpus separately. 
Then, using a weighting procedure, we merge all \glspl{LM}  into a single \glsxtrshort{LM}, producing one \glsxtrshort{LM} for Vietnamese language.
Using the development text, interpolation weights can be determined by giving highest weight to the source language models that have the lowest perplexity on the specified development set.

Table \ref{table:LM_stats} demonstrates how the \glsxtrshort{LM} performs. 
Vietnamese \glsxtrshort{LM} achieves a \glsxtrfull{PPL} of 67 and a \glsxtrfull{OOV} rate of 0.1\% on dev set.
