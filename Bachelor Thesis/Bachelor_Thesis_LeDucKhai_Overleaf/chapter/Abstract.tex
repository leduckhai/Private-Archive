\chapter{Abstract}
\label{ch: Abstract}

In today's interconnected globe, moving abroad is more and more prevalent, whether it's for employment, refugee resettlement, or other causes.
Language difficulties between natives and immigrants present a common issue on a daily basis, especially in medical domain.
This can make it difficult for patients and doctors to communicate during anamnesis or in the emergency room, which compromises patient care. 
The goal of the HYKIST Project is to develop a speech translation system to support patient-doctor communication with \glsxtrshort{ASR} and \glsxtrshort{MT}.

\glsxtrshort{ASR} systems have recently displayed astounding performance on particular tasks for which enough quantities of training data are available, such as LibriSpeech \cite{panayotov2015librispeech}.
Building a good model is still difficult due to a variety of speaking styles, acoustic and recording settings, and a lack of in-domain training data.
In this thesis, we describe our efforts to construct \glsxtrshort{ASR} systems for a conversational telephone speech recognition task in the medical domain for Vietnamese language to assist emergency room contact between doctors and patients across linguistic barriers.
In order to enhance the system's performance, we investigate various training schedules and data combining strategies. We also examine how best to make use of the little data that is available. 
The use of publicly accessible models like \glsxtrshort{XLSR-53} \cite{xlsr53} is compared to the use of customized pre-trained models, and both supervised and unsupervised approaches are utilized using \glsxtrshort{Wav2vec 2.0} \cite{wav2vec2} as architecture.

\clearpage