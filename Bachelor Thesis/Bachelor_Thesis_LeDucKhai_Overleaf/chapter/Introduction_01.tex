\chapter{Introduction}
\label{ch: Introduction}

\section{HYKIST Project}
\label{sec: HYKIST Project}

Migration to foreign countries is becoming more common in our globally connected world, whether for work, refugee movements, or other reasons.
As a result, language barriers between locals and foreigners are a common daily issue.
It is commonly known that, speaking with patients when they arrive at the hospital is crucial to their care. 
In medical care, a lack of or incorrect communication leads to underuse and misuse of medical services, lower quality of care, an increased rate of treatment errors, ineffective preventive measures for patients, and medical staff dissatisfaction.
The doctors then inquire about the patient's problems as well as his or her medical history. 
However, there are currently 20.8 million immigrants in Germany, with up to 30\% having only basic German language skills\footnote{\href{https://www.apptek.com/news/germanys-federal-ministry-of-health-awards-hykist-project-to-apptek-to-equip-critical-care-with-artificial-intelligence-driven-automatic-speech-translation-technology}{https://www.apptek.com/news/germanys-federal-ministry-of-health-awards-hykist-project-to-apptek-to-equip-critical-care-with-artificial-intelligence-driven-automatic-speech-translation-technology}}. 
If doctors and patients do not speak the same language, information communication is severely constrained, which has a negative impact on the patients' care. 
In the event that no common language is available, doctors can contact Triaphon which provides translators to aid communication between the patient and the doctor. 
These bi-lingual interpreters then assist in communication between the patient and the doctor.

In the HYKIST scenario, the doctor talks German to the patient, who speaks only Arabic or Vietnamese. 
Meanwhile, German and Arabic, or German and Vietnamese, are the languages spoken by the interpreters. 
The interpreters are not professional translators, instead, they are volunteers who contribute their time to the translation. 
This is problematic because the interpreters may require time to look up unfamiliar words, such as medical termini, or they may make a mistake.

The ultimate goal of the HYKIST project is to facilitate doctor-patient communication in a growing number of languages with the help of \glsxtrshort{ASR} and \glsxtrshort{MT} in order to meet the robust medical domain requirements via following steps: 
The interpreter is summoned via the hospital phone, which has an audio sampling rate of 8 kHz. 
We then create manual annotations with helps of our native-speaker volunteers. 
We investigate the use of additional outside-the-domain data for training as well as unsupervised methods because gathering project-specific data is an expensive and time-consuming operation.

\glsxtrshort{ASR} and \glsxtrshort{MT} technologies are linked with a dialogue system for initial anamnesis and integrated into an existing telecommunications platform for this purpose. 
First and foremost, the project collects dialogues in Arabic, Vietnamese, and German, which serve as the foundation for the development of algorithms and applications. 
During the project, the first technical tests for the accuracy and quality of the automated translations are already being performed. 
Following that, the overall system must be tested in a pilot test with clinical application partners for the area of emergency admissions and initial anamnesis in acute situations, as well as evaluated in a final clinical study for user acceptance.

The partners in the HYKIST Project are Triaphon\footnote{\href{https://triaphon.org/}{https://triaphon.org/}}, Fraunhofer Focus\footnote{\href{https://www.fokus.fraunhofer.de/en}{https://www.fokus.fraunhofer.de/en}} and AppTek GmbH\footnote{\href{https://www.apptek.com}{https://www.apptek.com}}.

\pagebreak
