\section{Experimental results}

% Supervised baselines and Monolingual pretraining

\begin{frame}{Supervised baselines}
    % Please add the following required packages to your document preamble:
% \usepackage{multirow}
\begin{table}[!ht]
\captionsetup{font=Large}
\centering
\begin{adjustbox}{width=0.5\columnwidth,center}
\begin{tabular}{|l|c|c|} 
\hline
\multicolumn{1}{|c|}{\multirow{2}{*}{AM}} & \multicolumn{2}{c|}{WER [\%]}  \\ 
\cline{2-3}
\multicolumn{1}{|c|}{}                    & Hykist dev & Hykist test       \\ 
\hline
GMM                                       & 62.2       & 59.7              \\ 
\hline
BLSTM                                     & 32.9       & 38.4              \\ 
\hline
Transformer                               & 31.0       & 35.1              \\
\hline
\end{tabular}
\end{adjustbox}
\caption{\center{\Glspl{WER} [\%] for supervised-only baselines on Vietnamese HYKIST data \cite{luescher2022:hykist}. 
\\ Models are trained on the monolingual in-house data.}}
\label{table:supervised_baselines}
%\TODO{the caption is very small}
\end{table}




\end{frame}

\begin{frame}{Monolingual pretraining}
% \usepackage{multirow}

\begin{table}[!ht]
\captionsetup{font=Large}
\centering
\begin{tabular}{|c|c|c|c|c|c|} 
\hline
\multicolumn{3}{|c|}{Pre-training}                                           & Fine-tuning         & \multicolumn{2}{c|}{WER [\%]}  \\ 
\hline
Data                          & ~Hours                & Epochs               & Epochs              & Hykist dev & Hykist test       \\ 
\hline
None                          & -                     & None                 & \multirow{2}{*}{33} & 32.1       & 36.6              \\ 
\cline{1-3}\cline{5-6}
Viet. in-house (sup. pret.)   & \multirow{2}{*}{219}  & 33                   &                     & 31.9       & 36.9              \\ 
\cline{1-1}\cline{3-6}
Viet. in-house (unsup. pret.) &                       & 100                  & \multirow{4}{*}{26} & 31.4       & 33.4              \\ 
\cline{1-3}\cline{5-6}
Aug. Viet. in-house           & \multirow{3}{*}{1168} & \multirow{3}{*}{300} &                     & 31.0       & 32.3              \\ 
\cline{1-1}\cline{5-6}
Viet. YT                      &                       &                      &                     & 29.8       & 35.2              \\ 
\cline{1-1}\cline{5-6}
Viet. in-house + YT           &                       &                      &                     & 25.3       & 27.2              \\
\hline
\end{tabular}
\caption{\center{All fine-tunings use the \textit{Large}\textsubscript{1-8} architecture and are trained until full convergence on Vietnamese in-house data and the recognition is done on HYKIST. All pre-trainings have been done with \textbf{random initialization}. Pre-training data "None" in the 3rd row means fine-tuning from scratch with wav2vec 2.0 \textit{Large}\textsubscript{1-8} architecture. 
In the experiment "supervised pretraining" in 4th row, we mimic the learning rate curve of the experiment "unsupervised pretraining" in 5th row.
The rest pretraining schedules are unsupervised.
The chosen number of pretraining epochs is the best checkpoint.}}
\label{table:monoling_pretraining}
%\TODO{add info: how much hours of data for fine tuning. in caption should suffice}
%\TODO{split | Data (hours) | $/rightarrow$ | Data | Hours |. I think that would make the table better readable}
%\TODO{move HYKIST label to WER or into a separate row??? might save a bit space in the width?}
\end{table}
\end{frame}    

\begin{frame}{Monolingual pretraining}
\begin{itemize}
    \item WERs of from scratch (32.1\% and 36.6\%) and in-house supervised pretraining (31.9\% and 36.9\%) reduce to (31.4\% and 33.4\%) of in-house pretraining 
    \\ \textrightarrow \, 
    On \textbf{wav2vec 2.0} architecture, the unsupervised pretraining helps 
    %\TODO{we don't have a control experiment where we trained longer thatn 33 epochs without pre-training, right?}
    
    \item WERs reduce from 31.4\% and 33.4\% to 31.0\% and 32.3\% 
    \\ \textrightarrow \,
    Data augmentation for pretraining is helpful
    
    \item WERs of Youtube data (29.8\% and 35.2\%) vs. in-house data (31.4\% and 33.4\%) 
    \\ \textrightarrow \,
    Having more data is not always helpful, if the domain mismatch is larger and the data has less speakers
    
    \item WERs by 25.3\% and 27.2\% when combining the in-house and YT data (the best result using solely monolingual data)
    \\ \textrightarrow \,
    Diversity of domains and speakers is necessary
\end{itemize}
%\TODO{these points need to be discussed with the table in view. so you need to say these things when you are showing the table, and then you can give a summary on the following slide}
\end{frame}
