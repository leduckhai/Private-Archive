\chapter{Experimental results}
\label{ch: Experimental_results}


\section{Supervised baselines}

% Please add the following required packages to your document preamble:
% \usepackage{multirow}
\begin{table}[!ht]
\captionsetup{font=Large}
\centering
\begin{adjustbox}{width=0.5\columnwidth,center}
\begin{tabular}{|l|c|c|} 
\hline
\multicolumn{1}{|c|}{\multirow{2}{*}{AM}} & \multicolumn{2}{c|}{WER [\%]}  \\ 
\cline{2-3}
\multicolumn{1}{|c|}{}                    & Hykist dev & Hykist test       \\ 
\hline
GMM                                       & 62.2       & 59.7              \\ 
\hline
BLSTM                                     & 32.9       & 38.4              \\ 
\hline
Transformer                               & 31.0       & 35.1              \\
\hline
\end{tabular}
\end{adjustbox}
\caption{\center{\Glspl{WER} [\%] for supervised-only baselines on Vietnamese HYKIST data \cite{luescher2022:hykist}. 
\\ Models are trained on the monolingual in-house data.}}
\label{table:supervised_baselines}
%\TODO{the caption is very small}
\end{table}





The baseline for Vietnamese is trained using the relevant in-house 8kHz monolingual telephone speech data.
The performance of the baseline \glsxtrshort{ASR} systems is displayed in Table \ref{table:supervised_baselines}.
The intrinsic difficulty of the language and the data causes the systems to function differently.
We believe there are various causes for this.
Due to the natural flow of speakers, the Vietnamese transcriptions are hard to reach high quality.
Additionally, Vietnamese also incorporates accented speech which is even difficult for native speakers to fully understand. 
Furthermore, the accent mismatch between Vietnamese fine-tuning and recognition data is also a major factor to the degradation of performance.
Our Vietnamese in-house dataset has dominantly 2 native accents, Northern and Central Vietnamese, and a very small fraction of Southern Vietnamese native accent, while HYKIST, because of being a simulation dataset, has unique accents - foreign-born accents - from both interpreter and patient sides.

On the HYKIST data, switching from a \glsxtrshort{GMM}/\glsxtrshort{HMM} framework to a hybrid \glsxtrshort{HMM} framework with a \glsxtrshort{RNN}-\glsxtrshort{BLSTM} results in a reduction of \glsxtrshort{WER} from 62.2\% and 59.7\% to 32.9\% and 38.4\% on dev and test set respectively.
Besides, the \glsxtrshort{WER}s continue decreasing to 31.0\% and 35.1\% by replacing \glsxtrshort{BLSTM} with  \glsxtrshort{Transformer} encoder.