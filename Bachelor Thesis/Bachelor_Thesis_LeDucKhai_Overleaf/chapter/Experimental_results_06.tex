\section{Intermediate loss analysis}

\subsection{Studies on Intermediate Focal Loss design}

\begin{table}[!ht]
\centering
\begin{adjustbox}{width=0.8\columnwidth,center}
\begin{tabular}{|c|c|c|c|c|c|} 
\hline
\multicolumn{6}{|c|}{Viet. in-house~\textit{Large}\textsubscript{1-8}}                                                                              \\ 
\hline
Layer & Hykist dev             & Hykist test            & CV dev                 & CV test                & Vivos                   \\ 
\hline
None  & 30.4                   & 33.4                   & 16.4                   & 35.6                   & 31.3                    \\ 
\hline
2     & 29.4                   & 34.0                   & \textbf{15.5}          & 35.7                   & 30.0                    \\ 
\hline
4     & \textbf{\textbf{28.6}} & \textbf{\textbf{33.0}} & 15.8                   & \textbf{\textbf{34.5}} & \textbf{\textbf{29.6}}  \\ 
\hline
6     & 29.1                   & \textbf{33.0}          & 16.6                   & 34.1                   & 29.9                    \\ 
\hline
2,6   & 29.1                   & 34.1                   & 15.6                   & 35.5                   & 30.7                    \\ 
\hline
3,5   & 29.1                   & 33.3                   & 16.4                   & 35.0                   & 30.1                    \\ 
\hline
\multicolumn{6}{|c|}{Viet. in-house~\textit{Base}}                                                                                  \\ 
\hline
None  & 30.2                   & 33.3                   & 16.6                   & 35.4                   & 30.9                    \\ 
\hline
3     & \textbf{28.7}          & 33.4                   & 17.0                   & 35.5                   & \textbf{30.1}           \\ 
\hline
6     & 29.0                   & 33.0                   & \textbf{\textbf{15.9}} & \textbf{\textbf{34.4}} & 30.5                    \\ 
\hline
9     & 29.5                   & \textbf{32.6}          & 14.8                   & 34.4                   & 30.5                    \\ 
\hline
4,8   & 29.3                   & 33.5                   & 16.2                   & 35.0                   & 30.3                    \\
\hline
\end{tabular}
\end{adjustbox}
\caption{
    \glspl{WER} {[}\%{]} comparison of \glsxtrshort{IF Loss} on different layers between 2 architecture sizes: \textit{Base} and \textit{Large}\textsubscript{1-8}.
    All models are finetuned until full convergence on Vietnamese in-house data and recognized on HYKIST, CommonVoice and VIVOS dataset.
    Layer "None" means the baseline (no application of \glsxtrshort{IF Loss}).
    }
\label{WER_intLoss_multiple_layer}
\end{table}

In Table \ref{WER_intLoss_multiple_layer}, we study variants of putting \glsxtrshort{IF Loss} at different layers. 
For \textit{Large}\textsubscript{1-8} model, we observe the performance degradation when moving the single \glsxtrshort{IF Loss} to different layers, while this gives mix results for the \textit{Base} model.
\cite{lee2021intermediate} also reports the same behavior when using Intermediate CTC Loss on 12-layer, 24-layer and 48-layer models in a supervised-only scenario. 
When applying 2 intermediate layers, we meet the degradation of performance for both \textit{Base} and \textit{Large}\textsubscript{1-8} models. 
From the experimental results, we therefore conclude that: Single \glsxtrshort{IF Loss} in the middle network layer yields the best result among variants.


\subsection{On-off Regularization technique}

\begin{table}[!ht]
\centering
\begin{adjustbox}{width=\columnwidth,center}
\begin{tabular}{|c|c|c|c|c|c|c|} 
\hline
\multicolumn{3}{|c|}{Off reg.}          & \multirow{6}{*}{\begin{tabular}[c]{@{}c@{}}Continue fine-tuning\\\longrightarrow\\\end{tabular}} & \multicolumn{3}{c|}{On reg.}             \\ 
\cline{1-3}\cline{5-7}
~Epochs & Hykist dev    & Hykist test   &                                                                                       & ~Epochs & Hykist dev    & Hykist test    \\ 
\cline{1-3}\cline{5-7}
0       & -             & -             &                                                                                       & 33      & 33.0          & 38.8           \\ 
\cline{1-3}\cline{5-7}
3       & 45.6          & 48.6          &                                                                                       & 33      & \textbf{32.5} & \textbf{38.4}  \\ 
\cline{1-3}\cline{5-7}
7       & 43.4 & 45.9 &                                                                                       & 33      & 34.3          & 39.5           \\ 
\cline{1-3}\cline{5-7}
10      & 44.6          & 47.1          &                                                                                       & 33      & 34.1          & 39.5           \\
\hline
\end{tabular}
\end{adjustbox}
\caption{
    \glspl{WER} {[}\%{]} comparison of On-off Regularization technique over epochs for raw waveform from scratch training.
    "Off Regularization" stage means training without regularization techniques like Dropout, SpecAugment and \glsxtrshort{IF Loss}.
    After training for some first epochs, the learning rate is reset and the model is preloaded in the "On Regularization" stage (all regularization techniques are turned on).
    All models are then continued being finetuned until full convergence on Vietnamese in-house data and recognized on HYKIST dataset.
    The 3rd row (0 epoch for "Off Regularization") is the baseline.}
\label{on_off_regularization}
\end{table}

To better exploit the \glsxtrshort{IF Loss}, we introduce the "\textbf{On-off Regularization} technique". 
We experiment this technique for raw waveform from scratch training.
Experimental results in Table \ref{on_off_regularization} show that, if we train without any regularization techniques ("Off Regularization" stage) for the first 3 epochs and then reset the learning rate and continue training with all regularizations turned on ("On Regularization" stage), we achieve the \glsxtrshort{WER}s reduction from 33.0\% and 38.8\% to 32.5\% and 38.4\% on dev and test set respectively compared to the baseline. 

In the future work, we plan to apply the "On-off Regularization" technique to other pretraining schedules.

\subsection{Combination of L2 regularization and Intermediate Focal Loss}

% \usepackage[normalem]{ulem}
% \usepackage{multirow}


\begin{table}[!ht]
\captionsetup{font=Large}
\centering
\begin{adjustbox}{width=0.8\columnwidth,center}
\begin{tabular}{|c|c|c|c|c|c|c|} 
\hline
\multirow{2}{*}{Arch.}             & \multirow{2}{*}{Init.}            & \multicolumn{2}{c|}{Pre-training}                      & \multirow{2}{*}{Reg.} & \multicolumn{2}{c|}{WER [\%]}  \\ 
\cline{3-4}\cline{6-7}
                                   &                                   & Data                            & Hours                &                       & Hykist dev & Hykist test       \\ 
\hline
\multirow{4}{*}{\textit{Large}\textsubscript{1-8}} & \multirow{6}{*}{None}             & \multirow{2}{*}{Viet. in-house} & \multirow{2}{*}{219} & With IF               & 28.6       & 33.0              \\ 
\cline{5-7}
                                   &                                   &                                 &                      & With IF + L2          & 28.6       & 32.9              \\ 
\cline{3-7}
                                   &                                   & \multirow{2}{*}{None}           & \multirow{2}{*}{-}   & With IF               & 33.0       & 38.8              \\ 
\cline{5-7}
                                   &                                   &                                 &                      & With IF + L2          & 31.4       & 36.4              \\ 
\cline{1-1}\cline{3-7}
\multirow{2}{*}{\textit{Base}}     &                                   & \multirow{4}{*}{Viet. in-house} & \multirow{4}{*}{219} & With IF               & 29.0       & 33.0              \\ 
\cline{5-7}
                                   &                                   &                                 &                      & With IF + L2          & 28.8       & 32.3              \\ 
\cline{1-2}\cline{5-7}
\multirow{2}{*}{\textit{Large}\textsubscript{1-8}} & \multirow{2}{*}{\textit{XLSR-53}} &                                 &                      & With IF               & 24.7       & 29.1              \\ 
\cline{5-7}
                                   &                                   &                                 &                      & With IF + L2          & 24.3       & 29.2              \\
\hline
\end{tabular}
\end{adjustbox}
\caption{
    \center{WERs comparison of the L2 Regularization combination with IF Loss.}}
\label{table:L2_comb}
\end{table}

Due to time constraint and project requirement, we only tune L2 regularization values in favor of HYKIST data performance.
By using grid-search technique for right L2 value selection, we are able to reduce \glspl{WER} of multiple pretraining schedules as shown in Table \ref{table:L2_comb}.
Notable results are seen on raw waveform from scratch training, where \glspl{WER} reduce from 33.0\% and 38.8\% (only \glsxtrshort{IF Loss}) to 31.4\% and 36.4\% (combination with L2 regularization) on dev and test set respectively, that makes \glsxtrshort{WERR} 5.5\% in average.

We stick with default parameters for \glsxtrshort{ICE Loss} and \glsxtrshort{IF Loss} because \glsxtrshort{WER}s do not fluctuate significantly when choosing other parameters.
However, the right parameters for L2 regularization are chosen based on grid-search strategy and different parameters make the \glsxtrshort{WER}s vary greatly.
Therefore, we recommend the use of L2 regularization should be the last regularization effort in the entire regularization pipeline due to its higher sentitivity to \glsxtrshort{WER}s compared to \glsxtrshort{ICE Loss} and \glsxtrshort{IF Loss}. 

